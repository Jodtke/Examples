\section{C++ Attributes}
\label{sec:attributes}

OpenMP directives for C++ can also be specified with 
%the implementation-defined 
the \code{directive} extension for the C++11 standard \plc{attributes}.
%https://en.cppreference.com/w/cpp/language/attributes

The C++ example below shows two ways to parallelize a \code{for} loop using the \code{\#pragma} syntax.
The first pragma uses the combined \code{parallel}~\code{for} directive, and the second
applies the uncombined closely nested directives, \code{parallel} and \code{for}, directly to the same statement. 
These are labeled PRAG 1-3.

Using the attribute syntax, the same construct in PRAG 1
is applied two different ways in attribute form, as shown in the ATTR 1 and ATTR 2 sections.
In ATTR 1 the attribute syntax is used with the \code{omp ::} namespace form.
In ATTR 2 the attribute syntax is used with the \code{using omp :} namespace form.

Next, parallization is attempted by applying directives using two different syntaxes.
For ATTR 3 and PRAG 4, the loop parallelization will fail to compile because multiple directives that
apply to the same statement must all use either the attribute syntax or the pragma syntax.
The lines have been commented out and labeled INVALID.

While multiple attributes may be applied to the same statement,
compilation may fail if the ordering of the directive matters.
For the ATTR 4-5 loop parallelization, the \code{parallel} directive precedes 
the \code{for} directive, but the compiler may reorder consecutive attributes.
If the directives are reversed, compilation will fail.

The attribute directive of the ATTR 6 section resolves the previous problem (in ATTR 4-5).
Here, the \code{sequence} attribute is used to apply ordering to the
directives of ATTR 4-5, using the \code{omp}~\code{::} namespace qualifier. (The
\code{using omp :} namespace form is not available for the \code{sequence} attribute.) 
Note, for the \code{sequence} attribute a comma must separate the \code{directive} extensions.


The last 3 pairs of sections (PRAG DECL 1-2, 3-4, and 5-6) show cases where 
directive ordering does not matter for \code{declare}~\code{simd} directives. 

In section PRAG DECL 1-2, the two loops use different SIMD forms of the \plc{P} function
(one with \code{simdlen(4)} and the other with \code{simdlen(8)}), 
as prescribed by the two different \code{declare}~\code{simd} directives
applied to the \plc{P} function definitions (at the beginning of the code). 
The directives use the pragma syntax, and order is not important. 
For the next set of loops 
(PRAG DECL 3-4) that use the \plc{Q} function, the attribute syntax is 
used for the \code{declare}~\code{simd} directives. 
The result is compliant code since directive order is irrelevant.
Sections ATTR DECL 5-6 are included for completeness. Here, the attribute 
form of the \code{simd} directive is used for loops calling the \plc{Q} function, 
in combination with the attribute form of the \code{declare}~\code{simd} 
directives declaring the variants for \plc{Q}.

\cppexample[5.0]{directive_syntax_attribute}{1}
