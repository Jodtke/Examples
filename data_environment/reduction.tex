\pagebreak

\section{Reduction}
\label{sec:reduction}

This section covers ways to perform reductions in parallel, task, taskloop, and SIMD regions.

\subsection{\code{reduction} Clause}
\label{subsec:reduction}

The following example demonstrates the \code{reduction} clause; note that some 
reductions can be expressed in the loop in several ways, as shown for the \code{max} 
and \code{min} reductions below:

\cexample[3.1]{reduction}{1}

\pagebreak

\ffreeexample{reduction}{1}

A common implementation of the preceding example is to treat it as if it had been 
written as follows:

\cexample{reduction}{2}

\fortranspecificstart
\ffreenexample{reduction}{2}

The following program is non-conforming because the reduction is on the 
\emph{intrinsic procedure name} \code{MAX} but that name has been redefined to be the variable 
named \code{MAX}.

\ffreenexample{reduction}{3}
% blue line floater at top of this page for "Fortran, cont."
\begin{figure}[t!]
\linewitharrows{-1}{dashed}{Fortran (cont.)}{8em}
\end{figure}

The following conforming program performs the reduction using the 
\emph{intrinsic procedure name} \code{MAX} even though the intrinsic \code{MAX} has been renamed 
to \code{REN}.

\ffreenexample{reduction}{4}

The following conforming program performs the reduction using 
\plc{intrinsic procedure name} \code{MAX} even though the intrinsic \code{MAX} has been renamed 
to \code{MIN}.

\ffreenexample{reduction}{5}
\fortranspecificend

%\pagebreak
The following example is non-conforming because the initialization (\code{a = 
0}) of the original list item \code{a} is not synchronized with the update of 
\code{a} as a result of the reduction computation in the \code{for} loop. Therefore, 
the example may print an incorrect value for \code{a}.

To avoid this problem, the initialization of the original list item \code{a} 
should complete before any update of \code{a} as a result of the \code{reduction} 
clause. This can be achieved by adding an explicit barrier after the assignment 
\code{a = 0}, or by enclosing the assignment \code{a = 0} in a \code{single} 
directive (which has an implied barrier), or by initializing \code{a} before 
the start of the \code{parallel} region.

\cexample[5.1]{reduction}{6}

\fexample[5.1]{reduction}{6}[1]

The following example demonstrates the reduction of array \plc{a}.  In C/C++ this is illustrated by the explicit use of an array section \plc{a[0:N]} in the \code{reduction} clause.  The corresponding Fortran example uses array syntax supported in the base language.  As of the OpenMP 4.5 specification the explicit use of array section in the \code{reduction} clause in Fortran is not permitted.  But this oversight has been fixed in the OpenMP 5.0 specification.


\cexample[4.5]{reduction}{7}

\ffreeexample{reduction}{7}

\subsection{Task Reduction}
\label{subsec:task_reduction}

In OpenMP 5.0 the \code{task\_reduction} clause was created for the \code{taskgroup} construct, 
to allow reductions among explicit tasks that have an \code{in\_reduction} clause.

In the \plc{task\_reduction.1} example below a reduction is performed as the algorithm
traverses a linked list. The reduction statement is assigned to be an explicit task using
a \code{task} construct and is specified to be a reduction participant with 
the \code{in\_reduction} clause.
A \code{taskgroup} construct encloses the tasks participating in the reduction, and
specifies, with the \code{task\_reduction} clause, that the taskgroup has tasks participating
in a reduction.  After the \code{taskgroup} region the original variable will contain 
the final value of the reduction.

Note: The \plc{res} variable is private in the \plc{linked\_list\_sum} routine
and is not required to be shared (as in the case of a \code{parallel} construct
reduction).


\cexample[5.0]{task_reduction}{1}

\ffreeexample[5.0]{task_reduction}{1}

In OpenMP 5.0 the \code{task} \plc{reduction-modifier} for the \code{reduction} clause was
introduced to provide a means of performing reductions among implicit and explicit tasks.

The \code{reduction} clause of a \code{parallel} or worksharing construct may
specify the \code{task} \plc{reduction-modifier} to include explicit task reductions
within their region, provided the reduction operators (\plc{reduction-identifiers})
and variables (\plc{list items}) of the participating tasks match those of the
implicit tasks.

There are 2 reduction use cases (identified by USE CASE \#) in the \plc{task\_reduction.2} example below.  

In USE CASE 1 a \code{task} modifier in the \code{reduction} clause 
of the \code{parallel} construct is used to include the reductions of any 
participating tasks, those with an \code{in\_reduction} clause and matching 
\plc{reduction-identifiers} (\code{+}) and list items (\code{x}).  

Note, a \code{taskgroup} construct (with a \code{task\_reduction} clause) in not
necessary to scope the explicit task reduction (as seen in the example above). 
Hence, even without the implicit task reduction statement (without the C \code{x++\;}  
and Fortran \code{x=x+1} statements), the \code{task} \plc{reduction-modifier} 
in a \code{reduction} clause of the \code{parallel} construct
can be used to avoid having to create a \code{taskgroup} construct 
(and its \code{task\_reduction} clause) around the task generating structure.

In USE CASE 2 tasks participating in the reduction are within a
worksharing region (a parallel worksharing-loop construct).
Here, too, no \code{taskgroup} is required, and the \plc{reduction-identifier} (\code{+})
and list item (variable \code{x}) match as required.


\cexample[5.0]{task_reduction}{2}

\ffreeexample[5.0]{task_reduction}{2}


\subsection{Reduction on Combined Target Constructs}
\label{subsec:target_reduction}

When a \code{reduction} clause appears on a combined construct that combines 
a \code{target} construct with another construct, there is an implicit map 
of the list items with a \code{tofrom} map type for the \code{target} construct. 
Otherwise, the list items (if they are scalar variables) would be 
treated as firstprivate by default in the \code{target} construct, which 
is unlikely to provide the intended behavior since the result of the
reduction that is in the firstprivate variable would be discarded 
at the end of the \code{target} region.

In the following example, the use of the \code{reduction} clause on \code{sum1}
or \code{sum2} should, by default, result in an implicit \code{tofrom} map for
that variable. So long as neither \code{sum1} nor \code{sum2} were already
present on the device, the mapping behavior ensures the value for
\code{sum1} computed in the first \code{target} construct is used in the
second \code{target} construct.

\cexample[5.0]{target_reduction}{1}

\ffreeexample[5.0]{target_reduction}{1}
%\clearpage

In next example,  the variables \code{sum1} and \code{sum2} remain on the
device for the duration of the \code{target}~\code{data} region so that it is
their device copies that are updated by the reductions. Note the significance
of mapping \code{sum1} on the second \code{target} construct; otherwise, it
would be treated by default as firstprivate and the result computed for
\code{sum1} in the prior \code{target} region may not be used. Alternatively, a
\code{target}~\code{update} construct could be used between the two
\code{target} constructs to update the host version of \code{sum1} with the
value that is in the corresponding device version after the completion of the
first construct.

\cexample[5.0]{target_reduction}{2}

\ffreeexample[5.0]{target_reduction}{2}


\subsection{Task Reduction with Target Constructs}
\label{subsec:target_task_reduction}

The following examples illustrate how task reductions can apply to target tasks
that result from a \code{target} construct with the \code{in\_reduction}
clause. Here, the \code{in\_reduction} clause specifies that the target task
participates in the task reduction defined in the scope of the enclosing
\code{taskgroup} construct. Partial results from all tasks participating in the
task reduction will be combined (in some order) into the original variable
listed in the \code{task\_reduction} clause before exiting the \code{taskgroup}
region. 

\cexample[5.1]{target_task_reduction}{1}

\ffreeexample[5.1]{target_task_reduction}{1}[1]

In the next pair of examples, the task reduction is defined by a
\code{reduction} clause with the \code{task} modifier, rather than a
\code{task\_reduction} clause on a \code{taskgroup} construct. Again, the
partial results from the participating tasks will be combined in some order
into the original reduction variable, \code{sum}.

\cexample[5.0]{target_task_reduction}{2a}

\ffreeexample[5.0]{target_task_reduction}{2a}

Next, the \code{task} modifier is again used to define a task reduction over
participating tasks. This time, the participating tasks are a target task
resulting from a \code{target} construct with the \code{in\_reduction} clause,
and the implicit task (executing on the primary thread) that calls
\code{host\_compute}. As before, the partial results from these paricipating
tasks are combined in some order into the original reduction variable.

\cexample[5.1]{target_task_reduction}{2b}

\ffreeexample[5.1]{target_task_reduction}{2b}[1]


\subsection{Taskloop Reduction}
\label{subsec:taskloop_reduction}

In the OpenMP 5.0 Specification the \code{taskloop} construct
was extended to include the reductions.

The following two examples show how to implement a reduction over an array
using taskloop reduction in two different ways.
In the first
example we apply the \code{reduction} clause to the \code{taskloop} construct. As it was
explained above in the task reduction examples, a reduction over tasks is
divided in two components: the scope of the reduction, which is defined by a
\code{taskgroup} region, and the tasks that participate in the reduction. In this
example, the \code{reduction} clause defines both semantics. First, it specifies that
the implicit \code{taskgroup} region associated with the \code{taskloop} construct is the scope of the
reduction, and second, it defines all tasks created by the \code{taskloop} construct as
participants of the reduction. About the first property, it is important to note
that if we add the \code{nogroup} clause to the \code{taskloop} construct the code will be
nonconforming, basically because we have a set of tasks that participate in a
reduction that has not been defined.

\cexample[5.0]{taskloop_reduction}{1}
\ffreeexample[5.0]{taskloop_reduction}{1}

%In the second example, we are computing exactly the same
%value but we do it in a very different way. The first thing that we do in the
%\plc{array\_sum} function is to create a \code{taskgroup} region that defines the scope of a
%new reduction using the \code{task\_reduction} clause.
%After that, we specify that a task and also the tasks generated
%by a taskloop will participate in that reduction using the \code{in\_reduction} clause
%on the \code{task} and \code{taskloop} constructs, respectively. Note that
%we also added the \code{nogroup} clause to the \code{taskloop} construct. This is allowed
%because what we are expressing with the \code{in\_reduction} clause is different
%from what we were expressing with the \code{reduction} clause. In one case we specify
%that the generated tasks will participate in a previously declared reduction
%(\code{in\_reduction} clause) whereas in the other case we specify that we want to
%create a new reduction and also that all tasks generated by the taskloop will
%participate on it.

The second example computes exactly the same value as in the preceding\plc{taskloop\_reduction.1} code section,
but in a very different way.
First, in the \plc{array\_sum} function a \code{taskgroup} region is created 
that defines the scope of a new reduction using the \code{task\_reduction} clause.
After that, a task and also the tasks generated by a taskloop participate in 
that reduction by using the \code{in\_reduction} clause on the \code{task}
and \code{taskloop} constructs, respectively. 
Note that the \code{nogroup} clause was added to the \code{taskloop} construct.
This is allowed because what is expressed with the \code{in\_reduction} clause
is different from what is expressed with the \code{reduction} clause.
In one case the generated tasks are specified to participate in a previously 
declared reduction (\code{in\_reduction} clause) whereas in the other case
creation of a new reduction is specified and also that all tasks generated 
by the taskloop will participate on it.

\cexample[5.0]{taskloop_reduction}{2}
\ffreeexample[5.0]{taskloop_reduction}{2}
%\clearpage

In the OpenMP 5.0 Specification, \code{reduction} clauses for the
\code{taskloop}~\code{ simd} construct were also added. 

The examples below compare reductions for the \code{taskloop} and the \code{taskloop}~\code{simd} constructs.
These examples illustrate the use of \code{reduction} clauses within 
"stand-alone" \code{taskloop} constructs, and the use of \code{in\_reduction} clauses for tasks of taskloops to participate
with other reductions within the scope of a parallel region.

\textbf{taskloop reductions:}

In the \plc{taskloop reductions} section of the example below, 
\plc{taskloop 1} uses the \code{reduction} clause 
in a \code{taskloop} construct for a sum reduction, accumulated in \plc{asum}. 
The behavior is as though a \code{taskgroup} construct encloses the 
taskloop region with a \code{task\_reduction} clause, and each taskloop
task has an \code{in\_reduction} clause with the specifications 
of the \code{reduction} clause.
At the end of the taskloop region \plc{asum} contains the result of the reduction.

The next taskloop, \plc{taskloop 2}, illustrates the use of the 
\code{in\_reduction} clause to participate in a previously defined
reduction scope of a \code{parallel} construct.

The task reductions of \plc{task 2} and \plc{taskloop 2} are combined
across the \code{taskloop} construct and the single \code{task} construct, as specified
in the \code{reduction(task,}~\code{+:asum)} clause of the \code{parallel} construct.
At the end of the parallel region \plc{asum} contains the combined result of all reductions.

\textbf{taskloop simd reductions:}

Reductions for the \code{taskloop}~\code{simd} construct are shown in the second half of the code.
Since each component construct, \code{taskloop} and \code{simd}, 
can accept a reduction-type clause, the \code{taskloop}~\code{simd} construct
is a composite construct, and the specific application of the reduction clause is defined
within the \code{taskloop}~\code{simd} construct section of the OpenMP 5.0 Specification.
The code below illustrates use cases for these reductions.

In the \plc{taskloop simd reduction} section of the example below,
\plc{taskloop simd 3} uses the \code{reduction} clause 
in a \code{taskloop}~\code{simd} construct for a sum reduction within a loop.
For this case a \code{reduction} clause is used, as one would use 
for a \code{simd} construct.
The SIMD reductions of each task are combined, and the results of these tasks are further 
combined just as in the \code{taskloop} construct with the \code{reduction} clause for \plc{taskloop 1}.
At the end of the taskloop region \plc{asum} contains the combined result of all reductions.

If a \code{taskloop}~\code{simd} construct is to participate in a previously defined 
reduction scope, the reduction participation should be specified with
a \code{in\_reduction} clause, as shown in the \code{parallel} region enclosing
\plc{task 4} and \plc{taskloop simd 4} code sections.  

Here the \code{taskloop}~\code{simd} construct's 
\code{in\_reduction} clause specifies participation of the construct's tasks as 
a task reduction within the scope of the parallel region.  
That is, the results of each task of the \code{taskloop} construct component 
contribute to the reduction in a broader level, just as in \plc{parallel reduction a} code section above.
Also, each \code{simd}-component construct
occurs as if it has a \code{reduction} clause, and the
SIMD results of each task are combined as though to form a single result for
each task (that participates in the \code{in\_reduction} clause).
At the end of the parallel region \plc{asum} contains the combined result of all reductions.

%Just as in \plc{parallel reduction a} the
%\code{taskloop simd} construct reduction results are combined 
%with the \code{task} construct reduction results
%as specified by the \code{in\_reduction} clause of the \code{task} construct
%and the \plc{task} reduction-modifier of the \code{reduction} clause of 
%the \code{parallel} construct.
%At the end of the parallel region \plc{asum} contains the combined result of all reductions.


\cexample[5.1]{taskloop_simd_reduction}{1}

\ffreeexample[5.1]{taskloop_simd_reduction}{1}[1]


\subsection{Reduction with the \code{scope} Construct}
\label{subsec:reduction_scope}

The following example illustrates the use of the \code{scope} construct 
to perform a reduction in a \code{parallel} region. The case is useful for 
producing a reduction and accessing reduction variables inside a \code{parallel} region 
without using a worksharing-loop construct.

\cppexample[5.1]{scope_reduction}{1}
\clearpage

\ffreeexample[5.1]{scope_reduction}{1}

