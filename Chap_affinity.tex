\cchapter{OpenMP Affinity}{affinity}
\label{chap:openmp_affinity}

OpenMP Affinity consists of a \code{proc\_bind} policy (thread affinity policy) and a specification of
places (\texttt{"}location units\texttt{"} or \plc{processors} that may be cores, hardware
threads, sockets, etc.).  
OpenMP Affinity enables users to bind computations on specific places.
The placement will hold for the duration of the parallel region. 
However, the runtime is free to migrate the OpenMP threads 
to different cores (hardware threads, sockets, etc.) prescribed within a given place, 
if two or more cores (hardware threads, sockets, etc.) have been assigned to a given place.

Often the binding can be managed without resorting to explicitly setting places.
Without the specification of places in the \code{OMP\_PLACES} variable, 
the OpenMP runtime will distribute and bind threads using the entire range of processors for 
the OpenMP program, according to the \code{OMP\_PROC\_BIND} environment variable
or the \code{proc\_bind} clause.  When places are specified, the OMP runtime
binds threads to the places according to a default distribution policy, or
those specified in the \code{OMP\_PROC\_BIND} environment variable or the
\code{proc\_bind} clause.

In the OpenMP Specifications document a processor refers to an execution unit that
is enabled for an OpenMP thread to use.  A processor is a core when there is
no SMT (Simultaneous Multi-Threading) support or SMT is disabled.  When 
SMT is enabled, a processor is a hardware thread (HW-thread). (This is the
usual case; but actually, the execution unit is implementation defined.) Processor
numbers are numbered sequentially from 0 to the number of cores less one (without SMT), or
0 to the number HW-threads less one (with SMT). OpenMP places use the processor number to designate
binding locations (unless an \texttt{"}abstract name\texttt{"} is used.) 


The processors available to a process may be a subset of the system's
processors.  This restriction may be the result of a 
wrapper process controlling the execution (such as \code{numactl} on Linux systems), 
compiler options, library-specific environment variables, or default
kernel settings.  For instance, the execution of multiple MPI processes,
launched on a single compute node, will each have a subset of processors as
determined by the MPI launcher or set by MPI affinity environment 
variables for the MPI library.  %Forked threads within an MPI process
%(for a hybrid execution of MPI and OpenMP code) inherit the valid 
%processor set for execution from the parent process (the initial task region) 
%when a parallel region forks threads.  The binding policy set in 
%\code{OMP\_PROC\_BIND} or the \code{proc\_bind} clause will be applied to 
%the subset of processors available to \plc{the particular} MPI process.

%Also, setting an explicit list of processor numbers in the \code{OMP\_PLACES} 
%variable before an MPI launch (which involves more than one MPI process) will
%result in unspecified behavior (and doesn't make sense) because the set of 
%processors in the places list must not contain processors outside the subset 
%of processors for an MPI process. A separate \code{OMP\_PLACES} variable must
%be set for each MPI process, and is usually accomplished by launching a script 
%which sets \code{OMP\_PLACES} specifically for the MPI process. 

Threads of a team are positioned onto places in a compact manner, a 
scattered distribution, or onto the primary thread's place, by setting the 
\code{OMP\_PROC\_BIND} environment variable or the \code{proc\_bind} clause  to 
\code{close}, \code{spread}, or \code{primary} (\code{master} has been deprecated), respectively.  When 
\code{OMP\_PROC\_BIND} is set to FALSE no binding is enforced; and 
when the value is TRUE, the binding is implementation defined to 
a set of places in the \code{OMP\_PLACES} variable or to places 
defined by the implementation if the \code{OMP\_PLACES} variable 
is not set. 

The \code{OMP\_PLACES} variable can also be set to an abstract name 
(\code{threads}, \code{cores}, \code{sockets}) to specify that a place is
either a single hardware thread, a core, or a socket, respectively. 
This description of the \code{OMP\_PLACES} is most useful when the 
number of threads is equal to the number of hardware thread, cores
or sockets.  It can also be used with a \code{close} or \code{spread} 
distribution policy when the equality doesn't hold.


% We need an example of using sockets, cores and threads:

% case 1 cores:

%     Hyper-Threads on (2 hardware threads per core)
%     1 socket x 4 cores x 2 HW-threads
%   
%     export OMP_NUM_THREADS=4
%     export OMP_PLACES=threads
%     
%          core #      0    1    2    3
%     processor #     0,1  2,3  4,5  6,7  
%     thread #     0  * _  _ _  _ _  _ _   #mask for thread 0
%     thread #     1  _ _  * _  _ _  _ _   #mask for thread 1
%     thread #     2  _ _  _ _  * _  _ _   #mask for thread 2
%     thread #     3  _ _  _ _  _ _  * _   #mask for thread 3

% case 2 threads:
%   
%     Hyper-Threads on (2 hardware threads per core)
%     1 socket x 4 cores x 2 HW-threads
%    
%     export OMP_NUM_THREADS=4
%     export OMP_PLACES=cores
%     
%          core #      0    1    2    3
%     processor #     0,1  2,3  4,5  6,7  
%     thread #     0  * *  _ _  _ _  _ _   #mask for thread 0
%     thread #     1  _ _  * *  _ _  _ _   #mask for thread 1
%     thread #     2  _ _  _ _  * *  _ _   #mask for thread 2
%     thread #     3  _ _  _ _  _ _  * *   #mask for thread 3

% case 3 sockets:
%   
%     No Hyper-Threads
%     3 socket x 4 cores 
%     
%     export OMP_NUM_THREADS=3
%     export OMP_PLACES=sockets
%     
%        socket #        0         1          2
%     processor #     0,1,2,3   4,5,6,7   8,9,10,11
%     thread #     0  * * * *   _ _ _ _   _ _  _  _   #mask for thread 0
%     thread #     0  _ _ _ _   * * * *   _ _  _  _   #mask for thread 1
%     thread #     0  _ _ _ _   _ _ _ _   * *  *  *   #mask for thread 2


%===== Examples Sections =====
\pagebreak
\section{\code{proc\_bind} Clause}
\label{sec:affinity}

The following examples demonstrate how to use the \code{proc\_bind} clause to 
control the thread binding for a team of threads in a \code{parallel} region. 
The machine architecture is depicted in the figure below. It consists of two sockets, 
each equipped with a quad-core processor and configured to execute two hardware 
threads simultaneously on each core. These examples assume a contiguous core numbering 
starting from 0, such that the hardware threads 0,1 form the first physical core.

\ifpdf
%\begin{figure}[htbp]
\centerline{\includegraphics[width=3.8in,keepaspectratio=true]%
{figs/proc_bind_fig.pdf}}
%\end{figure}
\fi

The following equivalent place list declarations consist of eight places (which 
we designate as p0 to p7):

\code{OMP\_PLACES=\texttt{"}\{0,1\},\{2,3\},\{4,5\},\{6,7\},\{8,9\},\{10,11\},\{12,13\},\{14,15\}\texttt{"}}

or

\code{OMP\_PLACES=\texttt{"}\{0:2\}:8:2\texttt{"}}

\subsection{Spread Affinity Policy}
\label{subsec:affinity_spread}


The following example shows the result of the \code{spread} affinity policy on 
the partition list when the number of threads is less than or equal to the number 
of places in the parent's place partition, for the machine architecture depicted 
above. Note that the threads are bound to the first place of each subpartition.

\cexample[4.0]{affinity}{1}

\fexample[4.0]{affinity}{1}

It is unspecified on which place the primary thread is initially started. If the 
primary thread is initially started on p0, the following placement of threads will 
be applied in the parallel region:

\begin{compactitem}
\item thread 0 executes on p0 with the place partition p0,p1

\item thread 1 executes on p2 with the place partition p2,p3

\item thread 2 executes on p4 with the place partition p4,p5

\item thread 3 executes on p6 with the place partition p6,p7
\end{compactitem}


If the primary thread would initially be started on p2, the placement of threads 
and distribution of the place partition would be as follows:

\begin{compactitem}
\item thread 0 executes on p2 with the place partition p2,p3

\item thread 1 executes on p4 with the place partition p4,p5

\item thread 2 executes on p6 with the place partition p6,p7

\item thread 3 executes on p0 with the place partition p0,p1
\end{compactitem}

The following example illustrates the \code{spread} thread affinity policy when 
the number of threads is greater than the number of places in the parent's place 
partition.

Let \plc{T} be the number of threads in the team, and \plc{P} be the number of places in the 
parent's place partition. The first \plc{T/P} threads of the team (including the primary
thread) execute on the parent's place. The next \plc{T/P} threads execute on the next 
place in the place partition, and so on, with wrap around. 

\cexample[4.0]{affinity}{2}

\ffreeexample[4.0]{affinity}{2}

It is unspecified on which place the primary thread is initially started. If the 
primary thread is initially started on p0, the following placement of threads will 
be applied in the parallel region:

\begin{compactitem}
\item threads 0,1 execute on p0 with the place partition p0

\item threads 2,3 execute on p1 with the place partition p1

\item threads 4,5 execute on p2 with the place partition p2

\item threads 6,7 execute on p3 with the place partition p3

\item threads 8,9 execute on p4 with the place partition p4

\item threads 10,11 execute on p5 with the place partition p5

\item threads 12,13 execute on p6 with the place partition p6

\item threads 14,15 execute on p7 with the place partition p7
\end{compactitem}

If the primary thread would initially be started on p2, the placement of threads 
and distribution of the place partition would be as follows:

\begin{compactitem}
\item threads 0,1 execute on p2 with the place partition p2

\item threads 2,3 execute on p3 with the place partition p3

\item threads 4,5 execute on p4 with the place partition p4

\item threads 6,7 execute on p5 with the place partition p5

\item threads 8,9 execute on p6 with the place partition p6

\item threads 10,11 execute on p7 with the place partition p7

\item threads 12,13 execute on p0 with the place partition p0

\item threads 14,15 execute on p1 with the place partition p1
\end{compactitem}

\subsection{Close Affinity Policy}
\label{subsec:affinity_close}

The following example shows the result of the \code{close} affinity policy on 
the partition list when the number of threads is less than or equal to the number 
of places in parent's place partition, for the machine architecture depicted above. 
The place partition is not changed by the \code{close} policy.

\cexample[4.0]{affinity}{3}

\fexample[4.0]{affinity}{3}

It is unspecified on which place the primary thread is initially started. If the 
primary thread is initially started on p0, the following placement of threads will 
be applied in the \code{parallel} region:

\begin{compactitem}
\item thread 0 executes on p0 with the place partition p0-p7

\item thread 1 executes on p1 with the place partition p0-p7

\item thread 2 executes on p2 with the place partition p0-p7

\item thread 3 executes on p3 with the place partition p0-p7
\end{compactitem}

If the primary thread would initially be started on p2, the placement of threads 
and distribution of the place partition would be as follows:

\begin{compactitem}
\item thread 0 executes on p2 with the place partition p0-p7

\item thread 1 executes on p3 with the place partition p0-p7

\item thread 2 executes on p4 with the place partition p0-p7

\item thread 3 executes on p5 with the place partition p0-p7
\end{compactitem}

The following example illustrates the \code{close} thread affinity policy when 
the number of threads is greater than the number of places in the parent's place 
partition.

Let \plc{T} be the number of threads in the team, and \plc{P} be the number of places in the 
parent's place partition. The first \plc{T/P} threads of the team (including the primary
thread) execute on the parent's place. The next \plc{T/P} threads execute on the next 
place in the place partition, and so on, with wrap around. The place partition 
is not changed by the \code{close} policy.

\cexample[4.0]{affinity}{4}

\ffreeexample[4.0]{affinity}{4}

It is unspecified on which place the primary thread is initially started. If the 
primary thread is initially running on p0, the following placement of threads will 
be applied in the parallel region:

\begin{compactitem}
\item threads 0,1 execute on p0 with the place partition p0-p7

\item threads 2,3 execute on p1 with the place partition p0-p7

\item threads 4,5 execute on p2 with the place partition p0-p7

\item threads 6,7 execute on p3 with the place partition p0-p7

\item threads 8,9 execute on p4 with the place partition p0-p7

\item threads 10,11 execute on p5 with the place partition p0-p7

\item threads 12,13 execute on p6 with the place partition p0-p7

\item threads 14,15 execute on p7 with the place partition p0-p7
\end{compactitem}

If the primary thread would initially be started on p2, the placement of threads 
and distribution of the place partition would be as follows:

\begin{compactitem}
\item threads 0,1 execute on p2 with the place partition p0-p7

\item threads 2,3 execute on p3 with the place partition p0-p7

\item threads 4,5 execute on p4 with the place partition p0-p7

\item threads 6,7 execute on p5 with the place partition p0-p7

\item threads 8,9 execute on p6 with the place partition p0-p7

\item threads 10,11 execute on p7 with the place partition p0-p7

\item threads 12,13 execute on p0 with the place partition p0-p7

\item threads 14,15 execute on p1 with the place partition p0-p7
\end{compactitem}

\subsection{Primary Affinity Policy}
\label{subsec:affinity_primary}

The following example shows the result of the \code{primary} affinity policy on 
the partition list for the machine architecture depicted above. The place partition 
is not changed by the primary policy.

\cexample[4.0]{affinity}{5}

\fexample[4.0]{affinity}{5}[1]
\clearpage

It is unspecified on which place the primary thread is initially started. If the 
primary thread is initially running on p0, the following placement of threads will 
be applied in the parallel region:

\begin{compactitem}
\item threads 0-3 execute on p0 with the place partition p0-p7
\end{compactitem}

If the primary thread would initially be started on p2, the placement of threads 
and distribution of the place partition would be as follows:

\begin{compactitem}
\item threads 0-3 execute on p2 with the place partition p0-p7
\end{compactitem}



\section{Task Affinity}
\label{sec: task_affinity}

The next example illustrates the use of the \code{affinity}
clause with a \code{task} construct.
The variables in the \code{affinity} clause provide a
hint to the runtime that the task should execute
"close" to the physical storage location of the variables. For example,
on a two-socket platform with a local memory component
close to each processor socket, the runtime will attempt to
schedule the task execution on the socket where the storage is located.

Because the C/C++ code employs a pointer, an array section is used in
the \code{affinity} clause.
Fortran code can use an array reference to specify the storage, as
shown here.

Note, in the second task of the C/C++ code the \plc{B} pointer is declared
shared.  Otherwise, by default, it would be firstprivate since it is a local
variable, and would probably be saved for the second task before being assigned
a storage address by the first task.  Also, one might think it reasonable to use
the \code{affinity} clause \plc{affinity(B[:N])} on the second \code{task} construct.
However, the storage behind \plc{B} is created in the first task, and the 
array section reference may not be valid when the second task is generated.
The use of the \plc{A} array is sufficient for this case, because one
would expect the storage for \plc{A} and \plc{B} would be physically "close"
(as provided by the hint in the first task).

\cexample[5.0]{affinity}{6}

\ffreeexample[5.0]{affinity}{6}


\section{Affinity Display}
\label{sec:affinity_display}

The following examples illustrate ways to display thread affinity.
Automatic display of affinity can be invoked by setting
the \code{OMP\_DISPLAY\_AFFINITY} environment variable to \code{TRUE}.  
The format of the output can be customized by setting the 
\code{OMP\_AFFINITY\_FORMAT} environment variable to an appropriate string.  
Also, there are API calls for the user to display thread affinity 
at selected locations within code.

For the first example the environment variable \code{OMP\_DISPLAY\_AFFINITY} has been
set to \code{TRUE}, and execution occurs on an 8-core system with \code{OMP\_NUM\_THREADS} set to 8.

The affinity for the primary thread is reported through a call to the API
\code{omp\_display\_affinity()} routine. For default affinity settings
the report shows that the primary thread can execute on any of the cores. 
In the following parallel region the affinity for each of the team threads is reported
automatically since the \code{OMP\_DISPLAY\_AFFINITY} environment variable has been set
to \code{TRUE}.

These two reports are often useful (as in hybrid codes using both MPI and OpenMP) 
to observe the affinity (for an MPI task) before the parallel region,
and during an OpenMP parallel region. Note: the next parallel region uses the 
same number of threads as in the previous parallel region and affinities are 
not changed, so affinity is NOT reported.

In the last parallel region, the thread affinities are reported
because the thread affinity has changed.

\cexample[5.0]{affinity_display}{1}

\ffreeexample[5.0]{affinity_display}{1}


In the following example 2 threads are forked, and each executes on a socket. Next,
a nested parallel region runs half of the available threads on each socket.

These OpenMP environment variables have been set:

\begin{compactitem}
\item \code{OMP\_PROC\_BIND}="TRUE"
\item \code{OMP\_NUM\_THREADS}="2,4"
\item \code{OMP\_PLACES}="\{0,2,4,6\},\{1,3,5,7\}"
\item \code{OMP\_AFFINITY\_FORMAT}="nest\_level= \%L, parent\_thrd\_num= \%a, thrd\_num= \%n, thrd\_affinity= \%A"
\end{compactitem}

where the numbers correspond to core ids for the system. Note, \code{OMP\_DISPLAY\_AFFINITY} is not
set and is \code{FALSE} by default.  This example shows how to use API routines to
perform affinity display operations.

For each of the two first-level threads the \code{OMP\_PLACES} variable specifies
a place with all the core-ids of the socket (\{0,2,4,6\} for one thread and \{1,3,5,7\} for the other).
(As is sometimes the case in 2-socket systems, one socket may consist
of the even id numbers, while the other may have the odd id numbers.)  The affinities
are printed according to the \code{OMP\_AFFINITY\_FORMAT} format: providing
the parallel nesting level (\%L), the ancestor thread number (\%a), the thread number (\%n)
and the thread affinity (\%A). In the nested parallel region within the \plc{socket\_work} routine
the affinities for the threads on each socket are printed according to this format.

\cexample[5.0]{affinity_display}{2}

\ffreeexample[5.0]{affinity_display}{2}

The next example illustrates more details about affinity formatting.
First, the \code{omp\_get\_affininity\_format()} API routine is used to 
obtain the default format. The code checks to make sure the storage 
provides enough space to hold the format.  
Next, the \code{omp\_set\_affinity\_format()} API routine sets a user-defined 
format: \plc{host=\%20H thrd\_num=\%0.4n binds\_to=\%A}.   

The host, thread number and affinity fields are specified by \plc{\%20H}, 
\plc{\%0.4n} and \plc{\%A}: \plc{H}, \plc{n} and \plc{A} are single character "short names" 
for the host, thread\_num and thread\_affinity data to be printed,
with format sizes of \plc{20}, \plc{4}, and "size as needed". 
The period (.) indicates that the field is displayed right-justified (default is left-justified) 
and the "0" indicates that any unused space is to be prefixed with zeros 
(e.g. instead of "1", "0001" is displayed for the field size of 4).

%The period (.) indicates that the field is displayed left-justified and the "0" indicates 
%that leading zeros are to be added so that the total length for the display of this “n” (thread_num) field is 4.

%The period (\plc{.}) indicates right justified and \plc{0} leading zeros.  
%All other text in the format is just user narrative.

Within the parallel region the affinity for each thread is captured by 
\code{omp\_capture\_affinity()} into a buffer array with elements indexed 
by the thread number (\plc{thrd\_num}).
After the parallel region, the thread affinities are printed in thread-number order.

If the storage area in buffer is inadequate for holding the affinity
data, the stored affinity data is truncated.  
%The \plc{max} reduction on the required storage, returned by 
%\code{omp\_capture\_affinity} in \plc{nchars}, is used to report 
%possible truncation (if \plc{max\_req\_store}  >  \plc{buffer\_store}).
The maximum value for the number of characters (\plc{nchars}) returned by 
\code{omp\_capture\_affinity} is captured by the \code{reduction(max:max\_req\_store)}
clause and the \plc{if(nchars >= max\_req\_store) max\_req\_store=nchars} statement. 
It is used to report possible truncation (if \plc{max\_req\_store} > \plc{buffer\_store}).

\cexample[5.0]{affinity_display}{3}

\ffreeexample[5.0]{affinity_display}{3}


\section{Affinity Query Functions}
\label{sec: affinity_query}

In the example below a team of threads is generated on each socket of
the system, using nested parallelism. Several query functions are used
to gather information to support the creation of the teams and to obtain 
socket and thread numbers.

For proper execution of the code, the user must create a place partition, such that
each place is a listing of the core numbers for a socket. For example,
in a 2 socket system with 8 cores in each socket, and sequential numbering
in the socket for the core numbers, the \code{OMP\_PLACES} variable would be set
to "\{0:8\},\{8:8\}", using the place syntax \{\plc{lower\_bound}:\plc{length}:\plc{stride}\},
and the default stride of 1.

The code determines the number of sockets (\plc{n\_sockets})
using the \code{omp\_get\_num\_places()} query function.
In this example each place is constructed with a list of 
each socket's core numbers, hence the number of places is equal
to the number of sockets. 

The outer parallel region forms a team of threads, and each thread 
executes on a socket (place) because the \code{proc\_bind} clause uses 
\code{spread} in the outer \code{parallel} construct.
Next, in the \plc{socket\_init} function, an inner parallel region creates a team 
of threads equal to the number of elements (core numbers) from the place
of the parent thread. Because the outer \code{parallel} construct uses 
a \code{spread} affinity policy, each of its threads inherits a subpartition of 
the original partition.  Hence, the \code{omp\_get\_place\_num\_procs} query function
returns the number of elements (here procs = cores) in the subpartition of the thread.  
After each parent thread creates its nested parallel region on the section,
the socket number and thread number are reported.

Note: Portable tools like hwloc (Portable HardWare LOCality package), which support
many common operating systems, can be used to determine the configuration of a system.  
On some systems there are utilities, files or user guides that provide configuration
information.  For instance, the socket number and proc\_id's for a socket 
can be found in the /proc/cpuinfo text file on Linux systems.

\cexample[4.5]{affinity_query}{1}

\ffreeexample[4.5]{affinity_query}{1}



