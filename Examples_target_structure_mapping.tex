\pagebreak
\section{Structure mapping}
\label{sec:structure_mapping}


%%%%%%%%%%%%%%%%%%%%%%%%%%%%%%%%%%%%%%%%%%%%%%%%%%%%%%%%%%%%%%%%%%%%%%%%%%%%%%%%%
In the example below, only structure elements \plc{S.a}, \plc{S.b} and \plc{S.p} 
of the \plc{S} structure appear in \code{map} clauses of a \code{target} construct.
Only these components have corresponding variables and storage on the device.  
Hence, the large arrays, \plc{S.buffera} and \plc{S.bufferb}, and the \plc{S.x} component have no storage 
on the device and cannot be accessed.  

Also, since the pointer member \plc{S.p} is used in an array section of a 
\code{map} clause, the array storage of the array section on the device, 
\plc{S.p[:N]}, is \emph{attached} to the pointer member \plc{S.p} on the device.
Explicitly mapping the pointer member \plc{S.p} is optional in this case.

Note: The buffer arrays and the \plc{x} variable have been grouped together, so that
the components that will reside on the device are all together (without gaps).
This allows the runtime to optimize the transfer and the storage footprint on the device.

\cexample{target_struct_map}{1}


%%%%%%%%%%%%%%%%%%%%%%%%%%%%%%%%%%%%%%%%%%%%%%%%%%%%%%%%%%%%%%%%%%%%%%%%%%%%%%%%%
The following example is a slight modification of the above example for 
a C++ class.  In the member function \plc{SAXPY::driver} 
the array section \plc{p[:N]} is \emph{attached} to the pointer member \plc{p}
on the device.
 
\cppexample{target_struct_map}{2}

%%%%%%%%%%%%%%%%%%%%%%%%%%%%%%%%%%%%%%%%%%%%%%%%%%%%%%%%%%%%%%%%%%%%%%%%%%%%%%%%%

%In this example a pointer, \plc{p}, is mapped in a 
%\code{target}~\code{data} construct (\code{map(p)}) and remains 
%persistent throughout the \code{target}~\code{data} region. The address stored
%on the host is not assigned to the device pointer variable, and 
%the device value is not copied back to the host at the end of the
%region (for a pointer, it is as though \code{map(alloc:p}) is effectively
%used).  The array section, \plc{p[:N]}, is mapped on both \code{target}
%constructs, and the pointer \plc{p} on the device is attached at the
%beginning and detached at the end of the regions to the newly created
%array section on the device.
%
%Also, in the following example the global variable, \plc{a}, becomes 
%allocated when it is first used on the device in a \code{target} region, 
%and persists on the device for all target regions.  The value on the
%device and host may be different, as shown by the print statements.
%The values may be made consistent with the \code{update} construct,
%as shown in the \plc{declare\_target.3.c} and \plc{declare\_target.3.f90} 
%examples.
%
%\cexample{target_struct_map}{2}
