\cchapter{Tasking}{tasking}
\label{chap:tasking}

Tasking constructs provide units of work to a thread for execution.  
Worksharing constructs do this, too (e.g. \code{for}, \code{do}, 
\code{sections}, and \code{singles} constructs); 
but the work units are tightly controlled by an iteration limit and limited 
scheduling, or a limited number of \code{sections} or \code{single} regions. 
Worksharing was designed 
with \texttt{"}data parallel\texttt{"} computing in mind.  Tasking was designed for 
\texttt{"}task parallel\texttt{"} computing and often involves non-locality or irregularity 
in memory access.

The \code{task} construct can be used to execute work chunks: in a while loop; 
while traversing nodes in a list; at nodes in a tree graph; 
or in a normal loop (with a \code{taskloop} construct).  
Unlike the statically scheduled loop iterations of worksharing, a task is 
often enqueued, and then dequeued for execution by any of the threads of the
team within a parallel region. The generation of tasks can be from a single 
generating thread (creating sibling tasks), or from multiple generators
in a recursive graph tree traversals. 
%(creating a parent-descendents hierarchy of tasks, see example 4 and 7  below). 
A \code{taskloop} construct
bundles iterations of an associated loop into tasks, and provides 
similar controls found in the \code{task} construct.

Sibling tasks are synchronized by the \code{taskwait} construct, and tasks
and their descendent tasks can be synchronized by containing them in
a \code{taskgroup} region.  Ordered execution is accomplished by specifying
dependences with a \code{depend} clause. Also, priorities can be
specified as hints to the scheduler through a \code{priority} clause.

Various clauses can be used to manage and optimize task generation,
as well as reduce the overhead of execution and to relinquish 
control of threads for work balance and forward progress. 

Once a thread starts executing a task, it is the designated thread 
for executing the task to completion, even though it may leave the
execution at a scheduling point and return later.  The thread is tied
to the task.  Scheduling points can be introduced with the \code{taskyield}
construct.  With an \code{untied} clause any other thread is allowed to continue
the task.  An \code{if} clause with an expression that evaluates to \plc{false} 
results in an \emph{undeferred} task, which instructs the runtime to suspend
the generating task until the undeferred task completes its execution.
By including the data environment of the generating task into the generated task with the 
\code{mergeable} and \code{final} clauses, task generation overhead can be reduced.

A complete list of the tasking constructs and details of their clauses
can be found in the \plc{Tasking Constructs} chapter of the OpenMP Specifications,
in the \plc{OpenMP Application Programming Interface} section.


%===== Examples Sections =====
\pagebreak
\section{\code{task} and \code{taskwait} Constructs}
\label{sec:task_taskwait}

The following example shows how to traverse a tree-like structure using explicit 
tasks. Note that the \code{traverse} function should be called from within a 
parallel region for the different specified tasks to be executed in parallel. Also 
note that the tasks will be executed in no specified order because there are no 
synchronization directives. Thus, assuming that the traversal will be done in post 
order, as in the sequential code, is wrong.

\cexample[3.0]{tasking}{1}

\ffreeexample[3.0]{tasking}{1}

In the next example, we force a postorder traversal of the tree by adding a \code{taskwait} 
directive. Now, we can safely assume that the left and right sons have been executed 
before we process the current node.

\cexample[3.0]{tasking}{2}

\ffreeexample[3.0]{tasking}{2}

The following example demonstrates how to use the \code{task} construct to process 
elements of a linked list in parallel. The thread executing the \code{single} 
region generates all of the explicit tasks, which are then executed by the threads 
in the current team. The pointer \plc{p} is \code{firstprivate} by default 
on the \code{task} construct so it is not necessary to specify it in a \code{firstprivate} 
clause.

\cexample[3.0]{tasking}{3}

\ffreeexample[3.0]{tasking}{3}

The \code{fib()} function should be called from within a \code{parallel}  region 
for the different specified tasks to be executed in parallel. Also, only one thread 
of the \code{parallel} region should call \code{fib()} unless multiple concurrent 
Fibonacci computations are desired. 

\cexample[3.0]{tasking}{4}

\fexample[3.0]{tasking}{4}
\clearpage

Note: There are more efficient algorithms for computing Fibonacci numbers. This 
classic recursion algorithm is for illustrative purposes.

The following example demonstrates a way to generate a large number of tasks with 
one thread and execute them with the threads in the team. While generating these 
tasks, the implementation may reach its limit on unassigned tasks.  If it does, 
the implementation is allowed to cause the thread executing the task generating 
loop to suspend its task at the task scheduling point in the \code{task} directive, 
and start executing unassigned tasks.  Once the number of unassigned tasks is sufficiently 
low, the thread may resume execution of the task generating loop.

\cexample[3.0]{tasking}{5}

\fexample[3.0]{tasking}{5}

The following example is the same as the previous one, except that the tasks are 
generated in an untied task. While generating the tasks, the implementation may 
reach its limit on unassigned tasks. If it does, the implementation is allowed 
to cause the thread executing the task generating loop to suspend its task at the 
task scheduling point in the \code{task} directive, and start executing unassigned 
tasks.  If that thread begins execution of a task that takes a long time to complete, 
the other threads may complete all the other tasks before it is finished.

In this case, since the loop is in an untied task, any other thread is eligible 
to resume the task generating loop. In the previous examples, the other threads 
would be forced to idle until the generating thread finishes its long task, since 
the task generating loop was in a tied task.

\cexample[3.0]{tasking}{6}

\fexample[3.0]{tasking}{6}

The following two examples demonstrate how the scheduling rules illustrated in 
Section 2.11.3 of the OpenMP 4.0 specification affect the usage of 
\code{threadprivate} variables in tasks. A \code{threadprivate} 
variable can be modified by another task that is executed by the same thread. Thus, 
the value of a \code{threadprivate} variable cannot be assumed to be unchanged 
across a task scheduling point. In untied tasks, task scheduling points may be 
added in any place by the implementation.

A task switch may occur at a task scheduling point. A single thread may execute 
both of the task regions that modify \code{tp}. The parts of these task regions 
in which \code{tp} is modified may be executed in any order so the resulting 
value of \code{var} can be either 1 or 2.

\cexample[3.0]{tasking}{7}


\fexample[3.0]{tasking}{7}

In this example, scheduling constraints prohibit a thread in the team from executing 
a new task that modifies \code{tp}  while another such task region tied to the 
same thread is suspended. Therefore, the value written will persist across the 
task scheduling point.

\cexample[3.0]{tasking}{8}


\fexample[3.0]{tasking}{8}

The following two examples demonstrate how the scheduling rules illustrated in 
Section 2.11.3 of the OpenMP 4.0 specification affect the usage of locks 
and critical sections in tasks.  If a lock is held 
across a task scheduling point, no attempt should be made to acquire the same lock 
in any code that may be interleaved.  Otherwise, a deadlock is possible.

In the example below, suppose the thread executing task 1 defers task 2.  When 
it encounters the task scheduling point at task 3, it could suspend task 1 and 
begin task 2 which will result in a deadlock when it tries to enter critical region 
1.

\cexample[3.0]{tasking}{9}


\fexample[3.0]{tasking}{9}

In the following example, \code{lock} is held across a task scheduling point. 
 However, according to the scheduling restrictions, the executing thread can't 
begin executing one of the non-descendant tasks that also acquires \code{lock} before 
the task region is complete.  Therefore, no deadlock is possible.

\cexample[3.0]{tasking}{10}

\ffreeexample[3.0]{tasking}{10}
\clearpage

The following examples illustrate the use of the \code{mergeable} clause in the 
\code{task} construct. In this first example, the \code{task} construct has 
been annotated with the \code{mergeable}  clause. The addition of this clause 
allows the implementation to reuse the data environment (including the ICVs) of 
the parent task for the task inside \code{foo} if the task is included or undeferred. 
Thus, the result of the execution may differ depending on whether the task is merged 
or not. Therefore the mergeable clause needs to be used with caution. In this example, 
the use of the mergeable clause is safe. As \code{x} is a shared variable the 
outcome does not depend on whether or not the task is merged (that is, the task 
will always increment the same variable and will always compute the same value 
for \code{x}).

\cexample[3.1]{tasking}{11}

\ffreeexample[3.1]{tasking}{11}

This second example shows an incorrect use of the \code{mergeable} clause. In 
this example, the created task will access different instances of the variable 
\code{x} if the task is not merged, as \code{x} is \code{firstprivate}, but 
it will access the same variable \code{x} if the task is merged. As a result, 
the behavior of the program is unspecified and it can print two different values 
for \code{x} depending on the decisions taken by the implementation.

\cexample[3.1]{tasking}{12}

\ffreeexample[3.1]{tasking}{12}

The following example shows the use of the \code{final} clause and the \code{omp\_in\_final} 
API call in a recursive binary search program. To reduce overhead, once a certain 
depth of recursion is reached the program uses the \code{final} clause to create 
only included tasks, which allow additional optimizations.

The use of the \code{omp\_in\_final} API call allows programmers to optimize 
their code by specifying which parts of the program are not necessary when a task 
can create only included tasks (that is, the code is inside a \code{final} task). 
In this example, the use of a different state variable is not necessary so once 
the program reaches the part of the computation that is finalized and copying from 
the parent state to the new state is eliminated. The allocation of \code{new\_state} 
in the stack could also be avoided but it would make this example less clear. The 
\code{final} clause is most effective when used in conjunction with the \code{mergeable} 
clause since all tasks created in a \code{final} task region are included tasks 
that can be merged if the \code{mergeable} clause is present.

\cexample[3.1]{tasking}{13}

\ffreeexample[3.1]{tasking}{13}

The following example illustrates the difference between the \code{if}  and the 
\code{final} clauses. The \code{if} clause has a local effect. In the first 
nest of tasks, the one that has the \code{if}  clause will be undeferred but 
the task nested inside that task will not be affected by the \code{if} clause 
and will be created as usual. Alternatively, the \code{final} clause affects 
all \code{task} constructs in the \code{final} task region but not the \code{final} 
task itself. In the second nest of tasks, the nested tasks will be created as included 
tasks. Note also that the conditions for the \code{if} and \code{final} clauses 
are usually the opposite.

\cexample[3.1]{tasking}{14}

\ffreeexample[3.1]{tasking}{14}


\pagebreak
\section{Task Priority}
\label{sec:task_priority}



%\subsection{Task Priority}
%\label{subsec:task_priority}

In this example we compute arrays in a matrix through a \plc{compute\_array} routine.
Each task has a priority value equal to the value of the loop variable \plc{i} at the
moment of its creation. A higher priority on a task means that a task is a candidate
to run sooner.

The creation of tasks occurs in ascending order (according to the iteration space of
the loop) but a hint, by means of the \code{priority} clause, is provided to reverse
the execution order.

\cexample[4.5]{task_priority}{1}

\ffreeexample[4.5]{task_priority}{1}


\pagebreak
\section{Task Dependences}
\label{sec:task_depend}

\subsection{Flow Dependence}
\label{subsec:task_flow_depend}

This example shows a simple flow dependence using a \code{depend} 
clause on the \code{task} construct.

\cexample[4.0]{task_dep}{1}

\ffreeexample[4.0]{task_dep}{1}

The program will always print \texttt{"}x = 2\texttt{"}, because the \code{depend} 
clauses enforce the ordering of the tasks. If the \code{depend} clauses had been 
omitted, then the tasks could execute in any order and the program and the program 
would have a race condition.

\subsection{Anti-dependence}
\label{subsec:task_anti_depend}

This example shows an anti-dependence using the \code{depend} 
clause on the \code{task} construct.

\cexample[4.0]{task_dep}{2}

\ffreeexample[4.0]{task_dep}{2}

The program will always print \texttt{"}x = 1\texttt{"}, because the \code{depend} 
clauses enforce the ordering of the tasks. If the \code{depend} clauses had been 
omitted, then the tasks could execute in any order and the program would have a 
race condition.

\subsection{Output Dependence}
\label{subsec:task_out_depend}

This example shows an output dependence using the \code{depend} 
clause on the \code{task} construct.

\cexample[4.0]{task_dep}{3}

\ffreeexample[4.0]{task_dep}{3}

The program will always print \texttt{"}x = 2\texttt{"}, because the \code{depend} 
clauses enforce the ordering of the tasks. If the \code{depend} clauses had been 
omitted, then the tasks could execute in any order and the program would have a 
race condition.

\pagebreak
\subsection{Concurrent Execution with Dependences}
\label{subsec:task_concurrent_depend}

In this example we show potentially concurrent execution of tasks using multiple 
flow dependences expressed using the \code{depend} clause on the \code{task} 
construct.

\cexample[4.0]{task_dep}{4}

\ffreeexample[4.0]{task_dep}{4}

The last two tasks are dependent on the first task. However there is no dependence 
between the last two tasks, which may execute in any order (or concurrently if 
more than one thread is available). Thus, the possible outputs are \texttt{"}x 
+ 1 = 3. x + 2 = 4. \texttt{"} and \texttt{"}x + 2 = 4. x + 1 = 3. \texttt{"}. 
If the \code{depend} clauses had been omitted, then all of the tasks could execute 
in any order and the program would have a race condition.

\subsection{Matrix multiplication}
\label{subsec:task_matrix_mult}

This example shows a task-based blocked matrix multiplication. Matrices are of 
NxN elements, and the multiplication is implemented using blocks of BSxBS elements.

\cexample[4.0]{task_dep}{5}

\ffreeexample[4.0]{task_dep}{5}

\subsection{\code{taskwait} with Dependences}
\label{subsec:taskwait_depend}

In this subsection three examples illustrate how the
\code{depend} clause can be applied to a \code{taskwait} construct to make the
generating task wait for specific child tasks to complete. This is an OpenMP 5.0 feature.
 In the same manner that
dependences can order executions among child tasks with \code{depend} clauses on
\code{task} constructs, the generating task can be scheduled to wait on child tasks
at a \code{taskwait} before it can proceed.

Note: Since the \code{depend} clause on a \code{taskwait} construct relaxes the
default synchronization behavior (waiting for all children to finish), it is important to
realize that child tasks that are not predecessor tasks, as determined by the \code{depend}
clause of the \code{taskwait} construct, may be running concurrently while the
generating task is executing after the taskwait.

In the first example the generating task waits at the \code{taskwait} construct
for the completion of the first child task because a dependence on the first task
is produced by \plc{x} with an \code{in} dependence type within the \code{depend}
clause of the \code{taskwait} construct.
Immediately after the first \code{taskwait} construct it is safe to access the
\plc{x} variable by the generating task, as shown in the print statement.
There is no completion restraint on the second child task.
Hence, immediately after the first \code{taskwait} it is unsafe to access the
\plc{y} variable since the second child task may still be executing.
The second \code{taskwait} ensures that the second child task has completed; hence
it is safe to access the \plc{y} variable in the following print statement.

\cexample[5.0]{task_dep}{6}

\ffreeexample[5.0]{task_dep}{6}

In this example the first two tasks are serialized, because a dependence on
the first child is produced by \plc{x} with the \code{in} dependence type
in the \code{depend} clause of the second task.
However, the generating task at the first \code{taskwait} waits only on the
first child task to complete, because a dependence on only the first child task
is produced by \plc{x} with an \code{in} dependence type within the
\code{depend} clause of the \code{taskwait} construct.
The second \code{taskwait} (without a \code{depend} clause) is included
to guarantee completion of the second task before \plc{y} is accessed.
(While unnecessary, the \code{depend(inout:} \code{y)} clause on the  2nd child task is
included to illustrate how the child task dependences can be completely annotated
in a data-flow model.)


\cexample[5.0]{task_dep}{7}

\ffreeexample[5.0]{task_dep}{7}


This example is similar to the previous one, except the generating task is
directed to also wait for completion of the second task.

The \code{depend} clause of the \code{taskwait} construct now includes an
\code{in} dependence type for \plc{y}.  Hence the generating task must now
wait on completion of any child task having \plc{y} with an \code{out}
(here \code{inout}) dependence type in its \code{depend} clause.
So, the \code{depend} clause of the \code{taskwait} construct now constrains
the second task to complete at the \code{taskwait}, too.
%--both tasks must now complete execution at the \code{taskwait}. 
(This change makes the second \code{taskwait} of the previous example unnecessary--
it has been removed in this example.)

Note: While a taskwait construct ensures that all child tasks have completed; a depend clause on a taskwait
construct only waits for specific child tasks (prescribed by the dependence type and list
items in the \code{taskwait}'s \code{depend} clause).
This and the previous example illustrate the need to carefully determine
the dependence type of variables in the \code{taskwait} \code{depend} clause
when selecting child tasks that the generating task must wait on, so that its execution after the
taskwait does not produce race conditions on variables accessed by non-completed child tasks.

\cexample[5.0]{task_dep}{8}

\ffreeexample[5.0]{task_dep}{8}

\pagebreak
\subsection{Mutually Exclusive Execution with Dependences}
\label{subsec:task_dep_mutexinoutset}

In this example we show a series of tasks, including mutually exclusive
tasks, expressing dependences using the \code{depend} clause on the
\code{task} construct.

The program will always print~6. Tasks T1, T2 and T3 will be scheduled first,
in any order. Task T4 will be scheduled after tasks T1 and T2 are
completed. T5 will be scheduled after tasks T1 and T3 are completed. Due
to the \code{mutexinoutset} dependence type on \code{c}, T4 and T5 may be
scheduled in any order with respect to each other, but not at the same
time. Tasks T6 will be scheduled after both T4 and T5 are completed.

\cexample[5.0]{task_dep}{9}

\ffreeexample[5.0]{task_dep}{9}

The following example demonstrates a situation where the \code{mutexinoutset}
dependence type is advantageous. If \code{shortTaskB} completes
before \code{longTaskA}, the runtime can take advantage of this by
scheduling \code{longTaskBC} before \code{shortTaskAC}.

\cexample[5.0]{task_dep}{10}

\ffreeexample[5.0]{task_dep}{10}

\subsection{Multidependences Using Iterators}
\label{subsec:depend_iterator}

The following example uses an iterator to define a dynamic number of
dependences.

In the \code{single} construct of a parallel region a loop generates n tasks
and each task has an \code{out} dependence specified through an element of
the \plc{v} array.  This is followed by a single task that defines an \code{in}
dependence on each element of the array.  This is accomplished by
using the \code{iterator} modifier in the \code{depend} clause, supporting a dynamic number
of dependences (\plc{n} here).

The task for the \plc{print\_all\_elements} function is not executed until all dependences
prescribed (or registered) by the iterator are fulfilled; that is,
after all the tasks generated by the loop have completed.

Note, one cannot simply use an array section in the \code{depend} clause
of the second task construct because this would violate the \code{depend} clause restriction:

"List items used in \code{depend} clauses of the same task or sibling tasks
must indicate identical storage locations or disjoint storage locations".

In this case each of the loop tasks use a single disjoint (different storage)
element in their \code{depend} clause; however,
the array-section storage area prescribed in the commented directive is neither
identical nor disjoint to the storage prescibed by the elements of the
loop tasks.  The iterator overcomes this restriction by effectively
creating n disjoint storage areas.

\cexample[5.0]{task_dep}{11}

\ffreeexample[5.0]{task_dep}{11}

\subsection{Dependence for Undeferred Tasks}
\label{subsec:depend_undefer_task}

In the following example, we show that even if a task is undeferred as specified
by an \code{if} clause that evaluates to \plc{false}, task dependences are
still honored.

The \code{depend} clauses of the first and second explicit tasks specify that
the first task is completed before the second task.

The second explicit task has an \code{if} clause that evaluates to \plc{false}.
This means that the execution of the generating task (the implicit task of
the \code{single} region) must be suspended until the second explict task
is completed.
But, because of the dependence, the first explicit task must complete first,
then the second explicit task can execute and complete, and only then 
the generating task can resume to the print statement.
Thus, the program will always print "\texttt{x = 2}".

\cexample[4.0]{task_dep}{12}
\clearpage

\ffreeexample[4.0]{task_dep}{12}


\pagebreak
\section{Task Detachment}
\label{sec:task_detachment}

% if used, then generated task must be completed.
% No definition of a detachable task

The \code{detach} clause on a \code{task} construct provides a mechanism for an asynchronous
routine to be called within a task block, and for the routine's
callback to signal completion to the OpenMP runtime, through an 
event fulfillment, triggered by a call to the \code{omp\_fulfill\_event} routine.
When a \code{detach} clause is used on a task construct,
completion of the \emph{detachable} task occurs when the task's structured
block is completed AND an \plc{allow-completion} event is
fulfilled by a call to the \code{omp\_fulfill\_event} 
routine with the \plc{event-handle} argument.

The first example illustrates the basic components used in a detachable task.
The second example is a program that executes asynchronous IO, and illustrates 
methods that are also inherent in asynchronous messaging within MPI and asynchronous commands in 
streams within GPU codes.
Interfaces to asynchronous operations found in IO, MPI and GPU parallel computing platforms
and their programming models are not standardized. 

-------------------------

The first example creates a detachable task
that executes the asynchronous \plc{async\_work} routine,
passing the \plc{omp\_fulfill\_event} function and the (firstprivate) event handle
to the function. Here, the \code{omp\_fulfill\_event} function is
the ``callback'' function to be executed at the end of the \plc{async\_work} function's
asynchronous operations,
with the associated data, \plc{event}. 

\cexample[5.0]{task_detach}{1}

\ffreeexample[5.0]{task_detach}{1}
\clearpage

%ASYNCHRONOUS IO

In the following example, text data is written asynchronously to the file \plc{async\_data},
using POSIX asynchronous IO (aio). An aio ``control block'', \plc{cb}, is set up to
send a signal when IO is complete, and the \plc{sigaction} function registers
the signal action, a callback to \plc{callback\_aioSigHandler}.

The first task (TASK1) starts the asynchronous IO and runs as a detachable task.
The second and third tasks (TASK2 and TASK3) perform synchronous IO to stdout with print statements.
The difference between the two types of tasks is that the thread for TASK1 is freed for 
other execution within the parallel region, while the threads for TASK2 and TASK3 wait
on the (synchronous) IO to complete, and cannot perform other work while the 
operating system is performing the synchronous IO. 
The \code{if} clause ensures that the detachable task is launched 
and the call to the \splc{aio_write} function returns
before TASK2 and TASK3 are generated (while the async IO occurs in the ``background'' and eventually
executes the callback function).  The barrier at the end of the parallel region ensures that the
detachable task has completed.

\cexample[5.0]{task_detach}{2}


\pagebreak
\section{\code{taskgroup} Construct}
\label{sec:taskgroup}

In this example, tasks are grouped and synchronized using the \code{taskgroup} 
construct.

Initially, one task (the task executing the \code{start\_background\_work()} 
call) is created in the \code{parallel} region, and later a parallel tree traversal 
is started (the task executing the root of the recursive \code{compute\_tree()} 
calls). While synchronizing tasks at the end of each tree traversal, using the 
\code{taskgroup} construct ensures that the formerly started background task 
does not participate in the synchronization, and is left free to execute in parallel. 
This is opposed to the behavior of the \code{taskwait} construct, which would 
include the background tasks in the synchronization.

\cexample[4.0]{taskgroup}{1}

\ffreeexample[4.0]{taskgroup}{1}


\pagebreak
\section{\code{taskyield} Construct}
\label{sec:taskyield}

The following example illustrates the use of the \code{taskyield}  directive. 
The tasks in the example compute something useful and then do some computation 
that must be done in a critical region. By using \code{taskyield} when a task 
cannot get access to the \code{critical} region the implementation can suspend 
the current task and schedule some other task that can do something useful. 

\cexample[3.1]{taskyield}{1}

\ffreeexample[3.1]{taskyield}{1}


\pagebreak
\section{\code{taskloop} Construct}
\label{sec:taskloop}

The following example illustrates how to execute a long running task concurrently with tasks created
with a \code{taskloop} directive for a loop having unbalanced amounts of work for its iterations.

The \code{grainsize} clause specifies that each task is to execute at least 500 iterations of the loop. 

The \code{nogroup} clause removes the implicit taskgroup of the \code{taskloop} construct; the explicit \code{taskgroup} construct in the example ensures that the function is not exited before the long-running task and the loops have finished execution.

\cexample[4.5]{taskloop}{1}

\ffreeexample[4.5]{taskloop}{1}

%\clearpage

Because a \code{taskloop} construct encloses a loop, it is often incorrectly 
perceived as a worksharing construct (when it is directly nested in 
a \code{parallel} region).

While a worksharing construct distributes the loop iterations across all threads in a team,
the entire loop of a \code{taskloop} construct is executed by every thread of the team.

In the example below the first taskloop occurs closely nested within 
a \code{parallel} region and the entire loop is executed by each of the \plc{T} threads; 
hence the reduction sum is executed \plc{T}*\plc{N} times.
 
The loop of the second taskloop is within a \code{single} region and is executed
by a single thread so that only \plc{N} reduction sums occur.  (The other
\plc{N}-1 threads of the \code{parallel} region will participate in executing the 
tasks. This is the common use case for the \code{taskloop} construct.)

In the example, the code thus prints \code{x1 = 16384} (\plc{T}*\plc{N}) and 
\code{x2 = 1024} (\plc{N}).

\cexample[4.5]{taskloop}{2}

\ffreeexample[4.5]{taskloop}{2}

\pagebreak
\section{Combined \code{parallel} \code{masked} and \code{taskloop} Constructs}
\label{sec:parallel_masked_taskloop}

Just as the \code{for} and \code{do} constructs were combined
with the \code{parallel} construct for convenience, so too, the combined
\code{parallel}~\code{masked}~\code{taskloop} and 
\code{parallel}~\code{masked}~\code{taskloop}~\code{simd}
constructs have been created for convenience when using the
\code{taskloop} construct.
  
In the following example the first \code{taskloop} construct is enclosed
by the usual \code{parallel} and \code{masked} constructs to form
a team of threads, and a single task generator (primary thread) for
the \code{taskloop} construct.

The same OpenMP operations for the first taskloop are accomplished by the second
taskloop with the \code{parallel}~\code{masked}~\code{taskloop} 
combined construct. 
The third taskloop uses the combined \code{parallel}~\code{masked}~\code{taskloop}~\code{simd} 
construct to accomplish the same behavior as closely nested \code{parallel masked},
and \code{taskloop simd} constructs.

As with any combined construct the clauses of the components may be used
with appropriate restrictions. The combination of the \code{parallel}~\code{masked} construct
with the \code{taskloop} or \code{taskloop}~\code{simd} construct produces no additional 
restrictions.

\cexample[5.1]{parallel_masked_taskloop}{1}

\ffreeexample[5.1]{parallel_masked_taskloop}{1}[1]


