\pagebreak
\section{\code{taskloop} Construct}
\label{sec:taskloop}

The following example illustrates how to execute a long running task concurrently with tasks created
with a \code{taskloop} directive for a loop having unbalanced amounts of work for its iterations.

The \code{grainsize} clause specifies that each task is to execute at least 500 iterations of the loop. 

The \code{nogroup} clause removes the implicit taskgroup of the \code{taskloop} construct; the explicit \code{taskgroup} construct in the example ensures that the function is not exited before the long-running task and the loops have finished execution.

\cexample[4.5]{taskloop}{1}

\ffreeexample[4.5]{taskloop}{1}

%\clearpage

Because a \code{taskloop} construct encloses a loop, it is often incorrectly 
perceived as a worksharing construct (when it is directly nested in 
a \code{parallel} region).

While a worksharing construct distributes the loop iterations across all threads in a team,
the entire loop of a \code{taskloop} construct is executed by every thread of the team.

In the example below the first taskloop occurs closely nested within 
a \code{parallel} region and the entire loop is executed by each of the \plc{T} threads; 
hence the reduction sum is executed \plc{T}*\plc{N} times.
 
The loop of the second taskloop is within a \code{single} region and is executed
by a single thread so that only \plc{N} reduction sums occur.  (The other
\plc{N}-1 threads of the \code{parallel} region will participate in executing the 
tasks. This is the common use case for the \code{taskloop} construct.)

In the example, the code thus prints \code{x1 = 16384} (\plc{T}*\plc{N}) and 
\code{x2 = 1024} (\plc{N}).

\cexample[4.5]{taskloop}{2}

\ffreeexample[4.5]{taskloop}{2}
