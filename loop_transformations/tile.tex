%\pagebreak
\section{\code{tile} Construct}
\label{sec:tile}

In the following example a \code{tile} construct transforms two nested loops
within the \texttt{func1} function into four nested loops.
The tile sizes in the \code{sizes} clause are applied from outermost
to innermost loops (left-to-right). The effective tiling operation is illustrated in
the \texttt{func2} function. 
(For easier illustration, tile sizes for all examples in this section evenly 
divide the iteration counts so that there are no remainders.)

In the following C/C++ code the inner loop traverses columns
and the outer loop traverses the rows of a 100x128 (row x column) matrix.  
The \code{sizes(5,16)} clause of the \code{tile} construct specifies
a 5x16 blocking, applied to the outer (row) and inner (column) loops.
The worksharing-loop construct before the \code{tile}
construct is applied after the transform.

\cexample[5.1]{tile}{1}

In the following Fortran code the inner loop traverses rows 
and the outer loop traverses the columns of a 128x100 (row x column) matrix.  
The  \code{sizes(5,16)} clause of the code{tile} construct specifies 
a 5x16 blocking, applied to the outer (column) and inner (row) loops.
The worksharing-loop construct before the \code{tile}
construct is applied after the transform.

\ffreeexample[5.1]{tile}{1}
\clearpage

This example illustrates transformation nesting.
Here, a 4x4 ``outer''  \code{tile} construct is applied to the ``inner'' tile transform shown in the example above.
The effect of the inner loop is shown in \texttt{func2} (cf.\ \texttt{func2} in tile.1.c).
The outer \code{tile} construct's \code{sizes(4,4)} clause applies a 4x4 tile upon the resulting
blocks of the inner transform.  The effective looping is shown in \texttt{func3}.

\cexample[5.1]{tile}{2}
\ffreeexample[5.1]{tile}{2}
