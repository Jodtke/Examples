
\pagebreak
\section{OpenMP Memory Model}
\label{sec:mem_model}

The following examples illustrate two major concerns for concurrent thread
execution: ordering of thread execution and memory accesses that may or may not
lead to race conditions.

In the following example, at Print 1, the value of \code{xval} could be either 2
or 5, depending on the timing of the threads. The \code{atomic} directives are
necessary for the accesses to \code{x} by threads 1 and 2 to avoid a data race.
If the atomic write completes before the atomic read, thread 1 is guaranteed to
see 5 in \code{xval}. Otherwise, thread 1 is guaranteed to see 2 in \code{xval}.

The barrier after Print 1 contains implicit flushes on all threads, as well as
a thread synchronization, so the programmer is guaranteed that the value 5 will
be printed by both Print 2 and Print 3. Since neither Print 2 or Print 3 are modifying
\code{x}, they may concurrently access \code{x} without requiring \code{atomic}
directives to avoid a data race.

\cexample[3.1]{mem_model}{1}

\ffreeexample[3.1]{mem_model}{1}

\pagebreak
The following example demonstrates why synchronization is difficult to perform
correctly through variables. The write to \code{flag} on thread 0 and the read
from \code{flag} in the loop on thread 1 must be atomic to avoid a data race.
When thread 1 breaks out of the loop, \code{flag} will have the value of 1.
However, \code{data} will still be undefined at the first print statement. Only
after the flush of both \code{flag} and \code{data} after the first print
statement will \code{data} have the well-defined value of 42.

\cexample[3.1]{mem_model}{2}

\fexample[3.1]{mem_model}{2}

\pagebreak
The next example demonstrates why synchronization is difficult to perform
correctly through variables. As in the preceding example, the updates to
\code{flag} and the reading of \code{flag} in the loops on threads 1 and 2 are
performed atomically to avoid data races on \code{flag}. However, the code still
contains data race due to the incorrect use of ``flush with a list'' after the
assignment to \code{data1} on thread 1. By not including \code{flag} in the
flush-set of that \code{flush} directive, the assignment can be reordered with
respect to the subsequent atomic update to \code{flag}. Consequentially,
\code{data1} is undefined at the print statement on thread 2.

\cexample[3.1]{mem_model}{3}

\fexample[3.1]{mem_model}{3}


