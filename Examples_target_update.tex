\pagebreak
\section{\code{target} \code{update} Construct}
\label{sec:target_update}

\subsection{Simple \code{target} \code{data} and \code{target} \code{update} Constructs}
\label{subsec:target_data_and_update}

The following example shows how the \code{target} \code{update} construct updates 
variables in a device data environment.

The \code{target} \code{data} construct maps array sections \plc{v1[:N]} and \plc{v2[:N]} 
(arrays \plc{v1} and \plc{v2} in the Fortran code) into a device data environment.

The task executing on the host device encounters the first \code{target} region 
and waits for the completion of the region.

After the execution of the first \code{target} region, the task executing on 
the host device then assigns new values to \plc{v1[:N]} and \plc{v2[:N]} (\plc{v1} and \plc{v2} arrays 
in Fortran code) in the task's data environment by calling the function \code{init\_again()}.

The \code{target} \code{update} construct assigns the new values of \plc{v1} and 
\plc{v2} from the task's data environment to the corresponding mapped array sections 
in the device data environment of the \code{target} \code{data} construct.

The task executing on the host device then encounters the second \code{target} 
region and waits for the completion of the region.

The second \code{target} region uses the updated values of \plc{v1[:N]} and \plc{v2[:N]}.

\cexample[4.0]{target_update}{1}

\ffreeexample[4.0]{target_update}{1}

\subsection{\code{target} \code{update} Construct with \code{if} Clause}
\label{subsec:target_update_if}

The following example shows how the \code{target} \code{update} construct updates 
variables in a device data environment.

The \code{target} \code{data} construct maps array sections \plc{v1[:N]} and \plc{v2[:N]} 
(arrays \plc{v1} and \plc{v2} in the Fortran code) into a device data environment. In between 
the two \code{target} regions, the task executing on the host device conditionally 
assigns new values to \plc{v1} and \plc{v2} in the task's data environment. The function \code{maybe\_init\_again()} 
returns \plc{true} if new data is written.

When the conditional expression (the return value of \code{maybe\_init\_again()}) in the 
\code{if} clause is \plc{true}, the \code{target} \code{update} construct 
assigns the new values of \plc{v1} and \plc{v2} from the task's data environment to the corresponding 
mapped array sections in the \code{target} \code{data} construct's device data 
environment.

\cexample[4.0]{target_update}{2}

\ffreeexample[4.0]{target_update}{2}

