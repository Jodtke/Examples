\cchapter{Memory Model}{memory_model}
\label{chap:memory_model}

OpenMP provides a shared-memory model that allows all threads on a given
device shared access to \emph{memory}. For a given OpenMP region that may be
executed by more than one thread or SIMD lane, variables in memory may be
\emph{shared} or \emph{private} with respect to those threads or SIMD lanes. A
variable's data-sharing attribute indicates whether it is shared (the
\emph{shared} attribute) or private (the \emph{private}, \emph{firstprivate},
\emph{lastprivate}, \emph{linear}, and \emph{reduction} attributes) in the data
environment of an OpenMP region. While private variables in an OpenMP region
are new copies of the original variable (with same name) that may then be
concurrently accessed or modified by their respective threads or SIMD lanes, a
shared variable in an OpenMP region is the same as the variable of the same
name in the enclosing region. Concurrent accesses or modifications to a
shared variable may therefore require synchronization to avoid data races.

OpenMP's memory model also includes a \emph{temporary view} of memory that is
associated with each thread. Two different threads may see different values for
a given variable in their respective temporary views. Threads may employ flush
operations for the purposes of making their temporary view of a variable
consistent with the value of the variable in memory. The effect of a given
flush operation is characterized by its flush properties -- some combination of
\emph{strong}, \emph{release}, and \emph{acquire} -- and, for \emph{strong}
flushes, a \emph{flush-set}. 

A \emph{strong} flush will force consistency between the temporary view and the
memory for all variables in its \emph{flush-set}.  Furthermore all strong flushes in a
program that have intersecting flush-sets will execute in some total order, and
within a thread strong flushes may not be reordered with respect to other
memory operations on variables in its flush-set. \emph{Release} and
\emph{acquire} flushes operate in pairs. A release flush may ``synchronize''
with an acquire flush, and when it does so the local memory operations that
precede the release flush will appear to have been completed before the local
memory operations on the same variables that follow the acquire flush.

Flush operations arise from explicit \code{flush} directives, implicit
\code{flush} directives, and also from the execution of \code{atomic}
constructs. The \code{flush} directive forces a  consistent view of local
variables of the thread executing the \code{flush}.  When a list is supplied on
the directive, only the items (variables) in the list are guaranteed to be
flushed.  Implied flushes exist at prescribed locations of certain constructs.
For the complete list of these locations and associated constructs, please
refer to the \plc{flush Construct} section of the OpenMP Specifications
document.

In this chapter, examples illustrate how race conditions may arise for accesses
to variables with a \plc{shared} data-sharing attribute when flush operations
are not properly employed.  A race condition can exist when two or more threads
are involved in accessing a variable and at least one of the accesses modifies
the variable.  In particular, a data race will arise when conflicting accesses
do not have a well-defined \emph{completion order}.  The existence of data
races in OpenMP programs result in undefined behavior, and so they should
generally be avoided for programs to be correct.  The completion order of
accesses to a shared variable is guaranteed in OpenMP through a set of memory
consistency rules that are described in the \plc{OpenMP Memory Consitency}
section of the OpenMP Specifications document.

%This chapter also includes examples that exhibit non-sequentially consistent
%(\emph{non-SC}) behavior. Sequential consistency (\emph{SC}) is the desirable
%property that the results of a multi-threaded program are as if all operations
%are performed in some total order, consistent with the program order of
%operations performed by each thread. OpenMP guarantees that a correct program
%(i.e. a program that does not have a data race) will exhibit SC behavior
%so long as the only \code{atomic} constructs it uses are SC atomic directives.



% The following table lists construct in which implied flushes exist, and the
% location of their execution.
% 
% %\begin{table}[hb]
% \begin{center}
% %\caption {Execution Location for Implicit Flushes. } 
% \begin{tabular}{ | p{0.6\linewidth} | l | } 
% \hline
% \code{CONSTRUCT}                                   & \makecell{\code{EXECUTION} \\ \code{LOCATION}} \\
% \hline
% \code{parallel}                                    & upon entry and exit \\
% \hline
% \makecell[l]{worksharing \\ \hspace{1.5em}\code{for}, \code{do} 
%                          \\ \hspace{1.5em}\code{sections} 
%                          \\ \hspace{1.5em}\code{single} 
%                          \\ \hspace{1.5em}\code{workshare} }  
%                                                    & upon exit \\ 
% \hline
% \code{critical}                                    & upon entry and exit \\
% \hline
% \code{target}                                      & upon entry and exit \\
% \hline
% \code{barrier}                                     & during \\
% \hline
% \code{atomic} operation with \plc{seq\_cst} clause & upon entry and exit \\
% \hline
% \code{ordered}*                                    & upon entry and exit \\
% \hline
% \code{cancel}** and \code{cancellation point}**    & during \\
% \hline
% \code{target data}                                 & upon entry and exit \\
% \hline
% \code{target update} + \code{to} clause,   
% \code{target enter data}                           & on entry \\
% \hline
% \code{target update} + \code{from} clause, 
% \code{target exit data}                            & on exit \\
% \hline
% \code{omp\_set\_lock}                              & during \\
% \hline
% \makecell[l]{ \code{omp\_set/unset\_lock}, \code{omp\_test\_lock}*** 
%            \\ \code{omp\_set/unset/test\_nest\_lock}*** }
%                                                    & during \\
% \hline
% task scheduling point                              & \makecell[l]{immediately \\ before and after} \\
% \hline
% \end{tabular}
% %\caption {Execution Location for Implicit Flushes. } 
% 
% \end{center}
% %\end{table}
% 
% * without clauses and with \code{threads} or \code{depend} clauses \newline
% ** when \plc{cancel-var} ICV is \plc{true} (cancellation is turned on) and cancellation is activated \newline
% *** if the region causes the lock to be set or unset
% 
% A flush with a list is implied for non-sequentially consistent \code{atomic} operations
% (\code{atomic} directive without a \code{seq\_cst} clause), where the list item is the
% specific storage location accessed atomically (specified as the \plc{x} variable
% in \plc{atomic Construct} subsection of the OpenMP Specifications document).

% Examples 1-3 show the difficulty of synchronizing threads through \code{flush} and \code{atomic} directives.


%===== Examples Sections =====

\pagebreak
\section{OpenMP Memory Model}
\label{sec:mem_model}

The following examples illustrate two major concerns for concurrent thread
execution: ordering of thread execution and memory accesses that may or may not
lead to race conditions.

In the following example, at Print 1, the value of \code{xval} could be either 2
or 5, depending on the timing of the threads. The \code{atomic} directives are
necessary for the accesses to \code{x} by threads 1 and 2 to avoid a data race.
If the atomic write completes before the atomic read, thread 1 is guaranteed to
see 5 in \code{xval}. Otherwise, thread 1 is guaranteed to see 2 in \code{xval}.

The barrier after Print 1 contains implicit flushes on all threads, as well as
a thread synchronization, so the programmer is guaranteed that the value 5 will
be printed by both Print 2 and Print 3. Since neither Print 2 or Print 3 are modifying
\code{x}, they may concurrently access \code{x} without requiring \code{atomic}
directives to avoid a data race.

\cexample[3.1]{mem_model}{1}

\ffreeexample[3.1]{mem_model}{1}

\pagebreak
The following example demonstrates why synchronization is difficult to perform
correctly through variables. The write to \code{flag} on thread 0 and the read
from \code{flag} in the loop on thread 1 must be atomic to avoid a data race.
When thread 1 breaks out of the loop, \code{flag} will have the value of 1.
However, \code{data} will still be undefined at the first print statement. Only
after the flush of both \code{flag} and \code{data} after the first print
statement will \code{data} have the well-defined value of 42.

\cexample[3.1]{mem_model}{2}

\fexample[3.1]{mem_model}{2}

\pagebreak
The next example demonstrates why synchronization is difficult to perform
correctly through variables. As in the preceding example, the updates to
\code{flag} and the reading of \code{flag} in the loops on threads 1 and 2 are
performed atomically to avoid data races on \code{flag}. However, the code still
contains data race due to the incorrect use of ``flush with a list'' after the
assignment to \code{data1} on thread 1. By not including \code{flag} in the
flush-set of that \code{flush} directive, the assignment can be reordered with
respect to the subsequent atomic update to \code{flag}. Consequentially,
\code{data1} is undefined at the print statement on thread 2.

\cexample[3.1]{mem_model}{3}

\fexample[3.1]{mem_model}{3}



\pagebreak
\section{Memory Allocators}
\label{sec:allocators}

OpenMP memory allocators can be used to allocate memory with
specific allocator traits.  In the following example an OpenMP allocator is used to
specify an alignment for arrays \plc{x} and \plc{y}. The
general approach for attributing traits to variables allocated by
OpenMP is to create or specify a pre-defined \plc{memory space}, create an
array of \plc{traits}, and then form an \plc{allocator} from the
memory space and trait.
The allocator is then specified
in an OpenMP allocation (using an API \plc{omp\_alloc()} function
for C/C++ code and an \code{allocate} directive for Fortran code
in the allocators.1 example).

In the example below the \plc{xy\_memspace} variable is declared
and assigned the default memory space (\plc{omp\_default\_mem\_space}).
Next, an array for \plc{traits} is created. Since only one
trait will be used, the array size is \plc{1}.
A trait is a structure in C/C++ and a derived type in Fortran,
containing 2 components: a key and a corresponding value (key-value pair).
The trait key used here is \plc{omp\_atk\_alignment} (an enum for C/C++
and a parameter for Fortran)
and the trait value of 64 is specified in the \plc{xy\_traits} declaration.
These declarations are followed by a call to the
\plc{omp\_init\_allocator()} function to combine the memory
space (\plc{xy\_memspace}) and the traits (\plc{xy\_traits})
to form an allocator (\plc{xy\_alloc}).

%In the C/C++ code the API  \plc{omp\_allocate()} function is used 
%to allocate space, similar to \plc{malloc}, except that the allocator 
%is specified as the second argument.  
%In Fortran an API allocation function is not available. 
%An \code{allocate} construct is used (with \plc{x} and \plc{y} 
%listed as the variables to be allocated), along
%with an \code{allocator} clause (specifying the \plc{xy\_alloc} as the allocator)
%for the following Fortran \plc{allocate} statement.

In the C/C++ code the API  \plc{omp\_allocate()} function is used
to allocate space, similar to \plc{malloc}, except that the allocator
is specified as the second argument.
In Fortran an \code{allocate} directive is used to specify an allocator
for a following Fortran \plc{allocate} statement.
A variable list may be supplied if the allocator
is to be applied to a subset of variables in the Fortran allocate
statement. Specifying the complete list is optional.
Here, the \plc{xy\_alloc} allocator is specified
in the \code{allocator} clause,
and the set of all variables used in the allocate statement is specified in the list.

%"for a following Fortran allocation statement" (no using "immediately" here)
% it looks like if you have a list, the allocation statement does not need
% to follow immediately.(?)
% spec5.0 157:19-20 The allocate directive must appear in the same scope as
% the declarations of each of its list items and must follow all such declarations.

%\pagebreak

\cexample[5.0]{allocators}{1}
\ffreeexample[5.0]{allocators}{1}



\pagebreak
\section{Race Conditions Caused by Implied Copies of Shared Variables in Fortran}
\fortranspecificstart
\label{sec:fort_race}

The following example contains a race condition, because the shared variable, which 
is an array section, is passed as an actual argument to a routine that has an assumed-size 
array as its dummy argument. The subroutine call passing an array section argument 
may cause the compiler to copy the argument into a temporary location prior to 
the call and copy from the temporary location into the original variable when the 
subroutine returns. This copying would cause races in the \code{parallel} region.

\ffreenexample{fort_race}{1}
\fortranspecificend




